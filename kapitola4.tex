\chapter{Testing}

\label{kap:testing} % id kapitoly pre prikaz ref
\paragraph*{}
In this chapter, we will display results generated using our tool and compare them with results from similar programs. We also discuss the source of data used for comparison as well as settings used by individual programs.

\section{Source of data}
\paragraph*{}
For our tests to have any informative value, we need to know a large amount of information about the samples used. Since we did not possess samples with sufficient information, we decided to simulate our samples. We simulate our data using InSilicoSeq, a tool designed to generate simulated sequenced reads from sequences input by the user. The reads can be generated using either one of pre-build error models or a custom model generated using a bam file with aligned reads. The pre-build error models include models for Illumina sequencers MiSeq, HiSeq and NovaSeq.

To generate our samples, we use all three default error models. As the input sequences, we use variable number of random genomes downloaded from NCBI. For our purposes, we used either ten or fifty genomes. We also adjust InSilicoSeq to generate either one or five million reads per sample. We have decided that using other number of reads or genomes is redundant and thus do not use them. With these settings, we generate a total of twelve paired-end input samples.

\section{Settings}
\paragraph*{}
Aside form Phendol, we proccess the generated samples using multiPhATE and RASTtk. In this section, we describe settings used for each tool.

\subsection{Phendol}
\paragraph*{}
We set Phendol to sample two million reads from the generated samples. This allows us to test Phendol on a full sized dataset in case of samples containing one million reads, as well as test in on a sampled dataset when working with samples containing five million reads. We used 32 threads to run SPAdes. Contigs generated by SPAdes were later used in analysis by other tools as well.

To test different setups of Phendol, we set the minimal percentage of identical matches in blast search to 75\% and 90\%. For the same reason, we set minimal percentage of coverage of sequences from endolysin database to 50\% and 75\%.

The minimal length of endolysins was left at the default value of 40 amino acids. Finally, we sorted the output by the percentage of the sequence covered by hits from database.

To reduce the runtime, we ran SPAdes only once per sample by manually copying its results into the working directories of other analyses. With these settings, we ran the pipeline four times per sample, resulting in the total of 48 analyses.

\subsection{multiPhATE}
\paragraph*{}
In the comparison, we use multiPhATE2, which is an enhanced version of multiPhATE. MultiPhATE uses a config file to specify all parameters. In the config file, we enabled the use of gene callers Phanotate, Prodigal and Glimmer. We also enabled the blastp search. MultiPhATE has two default databases against which it can run the blastp search. These are Prokaryotic Virus Orthologous Groups (pVOGs) \cite{grazziotin2016prokaryotic} and The Phage Annotation Tools and Methods (PhAnToMe) database. In our setup, we use both.

In the config file, we also set the list of genomes used in the analysis. In our case, the genomes set were the contigs produced by SPAdes during the run of Phendol. With the configuration complete, we ran multiPhATE once to get the analyses for every sample.

\subsection{RASTtk}
\paragraph*{}
Since we were unable to run the command line version of RASTtk, we used the online browser version.

For the \texttt{create-genome} script, we set the domain as virus and the genetic code to 11 Archaea, most Bacteria, most Virii, and some Mitochondria). The custom pipeline, we used for annotation, used the following scripts: \texttt{call-features-rRNA-SEED} to search for rRNA, \texttt{call-features-tRNA} to search for tRNA, \texttt{call-features-repeat-region-SEED} to find repeat regions with minimal identity of 95\% and minimal length of 100 base pairs, \texttt{call-selenoproteins} to find selenoproteins, \texttt{call-pyrrolysoproteins} to find pyrrolysoproteins, \texttt{call-features-crispr} to find CRISPR sequences, \texttt{call-features-CDS-glimmer3} and \texttt{call-features-CDS-prodigal} to find coding sequences using gene callers Glimmer3 and Prodigal, \texttt{annotate-proteins-kmer-v2}, \texttt{annotate-proteins-kmer-v1 -H}, \texttt{annotate-proteins-phage -H} and \texttt{annotate-proteins-similarity -H} to annotate proteins (option \texttt{-H} is used only for hypothetical proteins), \texttt{classify\_amr}, \texttt{annotate-special-proteins}, \texttt{annotate-families-figfam-v1}, \texttt{annotate-families\_patric}, \texttt{find-close-neighbors} and \texttt{annotate-strain-type-MLST}. Since the samples analysed only contain phages, it was unnecessary to use script \texttt{call-features-prophage-phispy}.

We used this pipeline to analyse the contigs created by SPAdes during the run of Phendol.

\section{Comparison}