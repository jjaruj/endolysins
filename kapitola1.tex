\chapter{Biological background}

\label{kap:background} % id kapitoly pre prikaz ref

In this chapter, we will explain terminology necessary for understanding our tool.

Bacteriophages or phages are a type of virus which evolved specifically to be able to infect bacteria. They are composed of a molecule of nucleic acid encased in a protein structure. While there are thousands of varieties of phages, each phage usually infects only one type or a few types of bacteria \cite{guttman2005basic}. This characteristic is utilized in phage therapy, which uses phages as an alternative to treatment using antibiotics. 
\paragraph*{}
There are two main advantages of phage therapy \cite{lin2017phage} when compared to antibiotics treatment. Antibiotics eliminate bacteria regardless of whether the bacteria is harmful to, beneficial for or does not affect the body. This in turn leads to damage of gut microbiota, which can create a change in bacterial metabolites, disrupt bacterial signaling and antimicrobial peptide secretion, damage regulation of function of gut immune cells Et al. \cite{zhang2019facing}. Unlike antibiotics, phage therapy targets only a specific type of bacteria allowing the body to maintain its health \cite{lin2017phage}. 
\paragraph*{}
Another advantage is derived from the fact, that unlike antibiotics, bacteriophages are alive and as such are subjected to evolution. Because bacteria are evolving, they are able to gain resistance to antibiotics when exposed to. When large variety of bacteria gain resistance to an antibiotic, the antibiotic is rendered ineffective. Similarly, when a bacteria gains resistance to multiple antibiotics, the treatment becomes even more difficult. Presently, many multiple antibiotics resistant bacteria already exist, like some strains of Staphylococus aureus or Mycobacterium tuberculosis \cite{guilfoile2007antibiotic}. As a result, it becomes necessary to develop a new antibiotic. However, this demands arduous research as well as substantial financial support. 
\paragraph*{}
In contrast, when a bacteria develops resistance to a particular strain of bacteriophage, the bacteriophage as a result of its evolution develops another method of infecting that bacteria, bypassing the resistance to the virus. Since this process happens naturally, uncovering new phages capable of infecting bacteria is significantly less demanding. It also implies an inexhaustible supply of treatment for bacterial infections, which is becoming increasingly more important with bacteria gradually developing resistances to an increasing number of antibiotics.
\paragraph*{}
In order for the phage to infect a bacteria, its tail fibres bind to specific receptors on the surface of the bacteria. While tail fibres and receptor pairing are highly specific, different types of phages might use the same receptors on membrane \cite{guttman2005basic}. The phage then creates a puncture in the bacterial membrane. Next, the nucleic acid is expelled from the phage through its tail and injected into the bacteria. When in cytoplasm, the viral genome in some cases becomes circular and resembles plasmid. After entering the bacteria, nucleic acid enters one of two cycles: lytic and lysogenic.
\paragraph*{}
In the lytic cycle, the viral nucleic acid is transcribed into messenger RNA (mRNA). If the viral genome consists of DNA, it is directly transcribed into mRNA. In case when it consists of RNA, it is first transcribed using an enzyme, reverse transcriptase, into DNA and then transcribed into mRNA. This mRNA utilizes cellular mechanisms of the host to destroy the hosts nucleic acid \cite{guttman2005basic}. After the destruction of the host nucleic acid, the viral genome is replicated and transcribed to produce proteins required for the assembly of new viruses. After enough proteins and viral genome is produced, they are assembled to create new bacteriophages. Next, the phage produces enzyme, endolysin, which causes the lysis of the cellular membrane. By destroying the membrane, newly formed bacteriophages are released, ending the lytic cycle.
\paragraph*{}
The lysogenic cycle differs from the lytic cycle by not immediately destroying the nucleic acid of the bacteria. Instead, it integrates its nucleic acid into the host genome, creating prophage. This is accomplished either by site-specific recombination or by random transposition \cite{guttman2005basic}. After integrating into the host genome, the prophage remains in a dormant state. The cellular mechanism of the bacteria remains unaffected by the prophage, so the bacteria continues its regular functions without alteration. During cell division, the prophage replicates with the host chromosomes resulting in the new bacterial cells already being infected by the phage. This process of replication is repeated until the conditions of the environment deteriorate. The deterioration can be induced by physical factors like UV radiation, low nutrient concentration or chemical factors. When the conditions deteriorate, the prophage might switch from lysogenic to the lytic cycle. 
\paragraph*{}
The lysogenic cycle of the phage has the advantage of increasing the amount of bacteriophages created from a single specimen. Since the phage is replicated during cell division along with the host cell, the number of proteins required to infect the same number of hosts is halved with each division, meaning the phage utilizing lysogenic cycle can reproduce in worse conditions than a phage only utilizing the lytic cycle.
\paragraph*{}
Because the purpose of phage therapy is the treatment of bacterial infection, the part of bacteriophages life cycle of most interest is the production of endolysin. Endolysins, alternatively termed phage lysins, are peptidoglycan hydrolases used by bacteriophages to enzymatically degrade the cellular membrane of the host bacteria, resulting in osmotic imbalance leading to rapid lysis of the membrane \cite{schmelcher2012bacteriophage}. The structure of lysins is in big part affected by whether the targeted bacteria is Gram-positive or Gram-negative, as the cellular membrane of these groups have different structure. 
\paragraph*{}
Endolysins of bacteriophages targeting Gram-positive bacteria have evolved in a way, where catalytic activity and substrate recognition are separated into two distinct varieties of functional domains, enzymatically active domains and cell wall binding domains \cite{schmelcher2012bacteriophage}. Enzymatically active domains impart the catalytic mechanism of lysin, the mechanism for cleaving of specific bonds in a cellular membrane of bacteria. The cell wall binding domain is responsible for targeting the lysin to its substrate and keeping it bound to parts of cell wall after cell lysis, reducing the probability of lysis of surrounding cells not yet infected by the phage.
\paragraph*{}
On the contrary, endolysins of phages targeting Gram-negative bacteria have a tendency to be small single-domain globular proteins without a specific domain for binding to the cell wall \cite{schmelcher2012bacteriophage}. Unlike endolysins from Gram-positive background, damage to surrounding bacteria in case of Gram-negative bacteria is prevented by its characteristic outer membrane, which protects the cell wall from the outside environment. Lysin is encased in the outer membrane, resulting in prevention of damage to other bacteria. It is surmised, that these endolysins fulfill their catalytic role more effectively as opposed to endolysins for Gram-positive bacteria \cite{schmelcher2012bacteriophage}, which are bound to one site on the membrane and as such reduced effectivity.
\paragraph*{}
The structure of endolysins is not limited to two domains, increasing the diversity of possible architectures. Most prominent structures include two N-terminal enzymatically active domains and one C-terminal cell wall binding domain, central cell wall binding domains separating two terminal enzymatically active domains, among others \cite{schmelcher2012bacteriophage}. Almost all currently described Gram-positive endolysins are encoded by single gene, simplifying their localization in the phage genome.
\paragraph*{}
Enzymatically active domains encompass the ability of an endolysin to catalyze a breakdown of the cell membrane. Based on the bond of the cell membrane an endolysin attacks, endolysins can be classified into five different groups: N-acetyl-\textbeta-D-muramidases (lysozymes) and lytic transglycosylases that cleave one of the glycosidic bonds of a sugar strand, N-acetyl-\textbeta-D-glucosaminidases cutting another glycosidic bond in the sugar strand, N-acetylmuramoyl-L-alanine amidases hydrolysing amide bond between sugar and peptide parts and endopeptidases cleaving the peptides making up interconnecting stem portion of the membrane \cite{schmelcher2012bacteriophage}. Any of these methods lead to destabilization and breakdown of the cell membrane.
\paragraph*{}
Cell wall binding domains allow an endolysin to recognise and bind (not using covalent bond) to ligands within cell membrane or other molecules associated with the cell wall. This significantly reduces range of activity for the enzymatically active domains. The spectrum of the cell wall binding domains can range from encompassing an entire genus of bacteria (lysostaphin domain targeting SH3b-like cell wall common to staphylococcal strains), making it generally broader than the host range of the particular phage, to the specificity of a single strain (endolysins of Listeria phage binding to groups of Listeria containing very specific ligands) \cite{schmelcher2012bacteriophage}.
\paragraph*{}
Our tool aims to find endolysins effective against particular bacteria. To ensure that the endolysin is indeed effective against the bacteria, we utilize the life cycle of bacteriophages. The tested type of bacteria is inserted into an environment containing a variety of phages. Phages, that are able to infect the bacteria do so while the rest remain outside of the bacteria. Once the infection is complete, infected bacteria are extracted from the environment. The phage genome inside the bacteria is in the form of a prophage or in the form of a cyclic or linear molecule of nucleic acid. After the extraction, the bacteria is forcibly killed and genetic material contained within is fragmented. Using various biochemical methods, nucleic acid containing bacterial genome is separated from viral nucleic acid. Precision of the separation is dependent on the method employed.
\paragraph*{}
What remains are fragments of phage nucleic acid of phages, that are capable of infecting the tested bacteria. This implies that endolysins encoded in the fragments are able to degrade the cell wall of the bacteria. The fragments are then sequenced by sequencing machines into reads. Our tool processes the reads and attempts to find the endolysins encoded in them.
